%-------------------------------------------------------------------------------
%	SECTION TITLE
%-------------------------------------------------------------------------------
\cvsection{Work Experience}


%-------------------------------------------------------------------------------
%	CONTENT
%-------------------------------------------------------------------------------
\begin{cventries}

%---------------------------------------------------------
  \cventry
    {Senior Software Engineer} % Job title
    {\href{https://diracinc.com/}{Dirac}} % Organisation
    {Manhattan, NY} % Location
    {August 2023 - Present} % Date(s)
    {
    \begin{cvitems} % Description(s) of tasks/responsibilities
        \item{\textbf{BuildOS:} Developed Python and C++ infrastructure for physics simulation of mechanical rigid body disassembly.}
        \item{Developed distributed microservices architecture using gRPC in Go, TypeScript, Python, and C++.}
        \item{Integrated a new CAD kernel to eliminate customer issues from STEP file import and enable colored output in the JavaScript front-end.}
        \item{Implemented continuous integration stages for code formatting, linting, unit testing, integration testing and deployment with GitHub actions.}
        \item{Contributed to developer and deployment workflows using Docker, Terraform, and Shell scripting.}
        \item{Maintained live servers with AWS Cloud infrastructure using Lambda, EC2, S3, and DynamoDB.}
      \end{cvitems}
    }
  \cventry
    {Senior Software Developer} % Job title
    {\href{https://nanotronics.co/}{Nanotronics}} % Organisation
    {Brooklyn, NY} % Location
    {June 2022 - June 2023} % Date(s)
    {
    \begin{cvitems} % Description(s) of tasks/responsibilities
        \item{\textbf{Modular Autoloader:} Developed flexible architecture for implementing new equipment front-end modules for semiconductor manufacturing.}
        \item{Wrote C++ abstractions for robots and sensors used in an AI defect analysis system for microscopic inspection of various substrate materials.}
        \item{Integrated sample-handling robots, external motors, pneumatic components, sensors, and other controls for a microscopic scanning system.}
        \item{Contributed to 4 new product releases with more than 12 new networked hardware components for handling of materials such as silicon wafers, semiconductor devices, glass and copper panels, and biological specimens.}
        \item{Supported our largest-yet microscope stage size of 650 mm x 650 mm, precisely controllable to 1 μm.}
        \item{Maintained a Windows MFC desktop application compliant with semiconductor industry communications standards in C++.}
      \end{cvitems}
    }
    \cventry
    {Robotics Engineer → Senior Software Engineer}
    {\href{https://rtr.ai/}{Realtime Robotics}}
    {Boston, MA}
    {July 2018 - February 2022}
    {
    \begin{cvitems}
        \item{\textbf{\href{https://rtr.ai/solutions/rapidplan/}{RapidPlan Create:}} Developed 2.0 version of the company's core robot motion-planning product which streamlined the user workflow and increased the modeling accuracy of our state-of-the-art collision checking technology.}
        \item{\textbf{\href{https://rtr.ai/solutions/rapidsense/}{RapidSense:}} Developed a GUI for performing extrinsic and intrinsic calibration of multiple RGBD cameras relative to a robotic arm to generate voxel images. Enabled visualization of voxel images accurate to the centimeter at a resolution of 128\textsuperscript{3} at 10Hz.}
        \item{\textbf{World Builder:} Developed a point-and-click robot workstation modeling prototype. Integrated differential evolution AI algorithm for optimizing robot workcells imported from 3\textsuperscript{rd} party applications. Curated a database of over 60 different robot models from 7 different OEMs.}
        \item{Implemented architectural improvements which scaled our system from supporting control of only up to 4 robots to control of up to 16 robots. Optimized memory size of saved project data by enabling implicit sharing on key classes. Achieved a 2-16x reduction in the number of user workflow steps for multi-robot work cells from the original version.}
        \item{Wrote reusable modules for hardware-acceleration of motion planning and collision checking algorithms on FPGAs achieving sub-millisecond motion plans for 6 degree-of-freedom robots.}
        \item{Wrote Python testing scripts for multi-robot control API ensuring process stability over week-long time spans for applications with up to 4 robots.}
        \item{Integrated AI object perception systems as grasp pose estimators for human-robot interaction and pick-and-place demonstrations with ROS, OpenCV, Python and C++.}
        \item{Created and maintained an internal GUI for simplifying the process of generating robot model files from CAD files used by both the quality assurance and applications engineering teams, implemented in Qt with C++.}
        \item{Created an internal GUI for testing motion planning and collision avoidance of up to 4 robots concurrently in Qt with Python.}
    \end{cvitems}
    }
\end{cventries}
